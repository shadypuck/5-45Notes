\documentclass[../notes.tex]{subfiles}

\pagestyle{main}
\renewcommand{\chaptermark}[1]{\markboth{\chaptername\ \thechapter\ (#1)}{}}
\setcounter{chapter}{7}

\begin{document}




\chapter{Exam 2}
\section{Exam 2 Review Sheet}
\begin{itemize}
    \item \marginnote{3/20:}Exam 1 content!
    \item Cross-coupling, revisited.
    \begin{itemize}
        \item Heck reaction.
        \begin{itemize}
            \item Mechanism: Oxidative additoin, ligand exchange, migratory insertion, $\beta$-hydride elimination, ligand exchange, reductive elimination.
            \item Regioselectivity: Balance of aryl to less-substituted \ce{C} (sterics), \ce{Pd} to $\delta^-$ \ce{C} (electronics).
            \begin{itemize}
                \item Triflates exaggerate $\delta^+$ on \ce{Pd}.
            \end{itemize}
        \end{itemize}
        \item Buchwald-Hartwig amination.
        \begin{itemize}
            \item Palladium source: \ce{Pd(OAc)2}, \ce{Pd2(dba)3}, precatalyst (e.g., the following).
            \begin{center}
                \vspace{0.5em}
                \footnotesize
                \chemfig{*6(=-=*6(-Pd(-[:-10]L)(-[:-50]OMs)-NHMe-)-(-*6(=-=-=-))=-)}
                \vspace{0.5em}
            \end{center}
            \begin{itemize}
                \item $\ce{L}=\text{anything not super bulky, e.g., \ce{{}^{\emph{t}}BuBrettPhos}.}$
            \end{itemize}
            \item Ligand: BINAP, Xantphos, many others.
            \item Base: Weak (\ce{Cs2CO3}) or strong (\ce{NaO{}^{\emph{t}}Bu}) can work.
            \item Solvent: Ethereal or aromatic hydrocarbon can work.
            \item Temperature: RT-\SI{140}{\celsius}.
            \item Mechanism: Activation then oxidative addition, binding, deprotonation and loss of \ce{X-} to give \ce{base*HX} salt, reductive elimination.
        \end{itemize}
        \item Wacker oxidation.
        \item Indole syntheses, promoted by Pd-catalyzed cross-coupling.
        \begin{itemize}
            \item New mechanisms, but do not create any new substitution schemes.
        \end{itemize}
        \item Ullmann/Goldberg couplings.
        \begin{itemize}
            \item Original method: Stoichiometric strong base, polar solvents, high temperatures.
            \item Modern method: Ligands.
            \begin{itemize}
                \item Heterocycle-amide couplings (Goldberg-type): Proline.
                \item Amine couplings (Ullmann-type): Oxalamides (even able heteroaryl chlorides!).
            \end{itemize}
            \item Mechanism: Nucleophile binding, oxidative addition, reductive elimination.
            \begin{itemize}
                \item May also be Pd-like in some cases (with oxidative addition first).
            \end{itemize}
            \item Use $sp^2$-bromides and iodides (\emph{not} $sp^2$-triflates).
            \item Catalytic \ce{CuI} helpful.
        \end{itemize}
    \end{itemize}
    \item New heterocycles and their key properties.
    \begin{itemize}
        \item 1,2,3-triazole.
        \begin{itemize}
            \item Stability: Up to \SI{500}{\celsius}.
            \item Tautomerization: Rapid among all when unsubstituted.
            \item Amphoteric, like imidazole.
            \item Acidity: Protonated form ($\pKa=1.2$).
            \item 1,2,3-triazole containing chemical: Benzotriazole (chemical photography).
        \end{itemize}
        \item 1,2,4-triazole.
        \begin{itemize}
            \item Tautomerization: Rapid among all when unsubstituted.
            \item Acidity: Protonated form ($\pKa=2.2$).
            \item 1,2,4-triazole containing drug: Epoxiconazole (fungicide).
        \end{itemize}
        \item Tetrazole.
        \begin{itemize}
            \item Tautomerization: Rapid.
            \item Acidity: Comparable to a carboxylic acid.
            \item Tetrazole-containing drug: Valsartan.
        \end{itemize}
        \item Oxazole.
        \begin{itemize}
            \item Basicity: Mildly basic nitrogen (not great because poor resonance).
            \item Acidity: C2 can be deprotonated with LDA.
            \item Aromaticity: Less aromatic than thiazole.
            \item Oxazole-containing drug: Neopeltolide (oncology).
        \end{itemize}
        \item Isoxazole.
        \begin{itemize}
            \item Reactivity: Some EAS.
            \item Isoxazole-containing drug: Valdecoxib (pain).
        \end{itemize}
        \item Thiazole.
        \begin{itemize}
            \item \emph{Reactivity, acidity, basicity, nucleophilicity, protonation, $\pi$-excessive/deficient, hydrogen bonding, tautomerization, etc.}
            \item Basicity: More basic than oxazole (lower EN of \ce{S} vs. \ce{O}).
            \item Aromaticity: Greater than oxazole.
            \item Reactivity: EAS at enamine carbon (C5).
            \item Thiazole-containing natural product: Thiamine aka vitamin B1.
        \end{itemize}
        \item 1,2,4-oxadiazole.
        \begin{itemize}
            \item Reactivity: Explosophore.
        \end{itemize}
        \item 1,3,4-oxadiazole.
        \begin{itemize}
            \item 1,3,4-oxadiazole-containing drug: Raltegravir (HIV).
        \end{itemize}
    \end{itemize}
    \item 1,2,3-triazole synthesis.
    \begin{itemize}
        \item $[3+2]$ dipolar azide-alkyne cycloadditions.
        \begin{figure}[h!]
            \centering
            \footnotesize
            \begin{subfigure}[b]{0.45\linewidth}
                \centering
                \schemestart
                    \chemfig{[2]~[2]-R'}
                    \arrow{0}[,0.1]\+{,,2.7em}
                    \chemfig[fixed length=false]{R-[:150]\charge{180=$\ominus$}{N}-[2]\charge{180=$\oplus$}{N}~[2]N-[4,0.5,,,opacity=0]}
                    \arrow{->[\ce{Cu}]}
                    \chemfig[fixed length=false]{*5(-N(-R)-N=N-(-R')=)}
                \schemestop
                \caption{CuAAC (1,4-substitution).}
            \end{subfigure}
            \begin{subfigure}[b]{0.45\linewidth}
                \centering
                \schemestart
                    \chemfig{R'-[2]~[2]}
                    \arrow{0}[,0.1]\+{,,2.7em}
                    \chemfig[fixed length=false]{R-[:150]\charge{180=$\ominus$}{N}-[2]\charge{180=$\oplus$}{N}~[2]N-[4,0.5,,,opacity=0]}
                    \arrow{->[\ce{Cp^*RuCl}]}[,1.4]
                    \chemfig[fixed length=false]{*5((-R')-N(-R)-N=N-=)}
                \schemestop
                \caption{RuAAC (1,5-substitution).}
            \end{subfigure}
        \end{figure}
        \begin{itemize}
            \item Making aliphatic azides: Use S\textsubscript{N}2.
            \item Making aryl azides: Use 1. HONO, 2. \ce{NaN3}.
            \item Possible mechanism: Sonogashira-type copper acetylide formation, azide coordination, electrocyclization, ring contraction, elimination.
        \end{itemize}
    \end{itemize}
    \item 1,2,4-triazole synthesis.
    \begin{itemize}
        \item \textbf{Paal-Knorr-type} (1,2,4-triazole synthesis): (3)(4)5-substitution.
        \begin{center}
            \footnotesize
            \schemestart
                \chemfig{R^5-(=[:-60]O)-[:60,,,2]HN-NH-[:-60,,1](=[:-120]O)-R^3}
                \+
                \chemfig{R^4-NH_2}
                \arrow{->[\ce{P4O10}][$\Delta$]}[,1.1]
                \chemfig[fixed length=false]{*5([:-18](-R^5)-N(-R^4)-(-R^3)=N-N=)}
            \schemestop
        \end{center}
        \item \emph{sym}-Triazine-type (1,2,4-triazole synthesis): 1-substitution.
        \begin{center}
            \footnotesize
            \schemestart
                \chemfig{*6(-N=-N=-N=)}
                \arrow{0}[,0.1]\+{,,-1.5em}
                \chemfig{R^1-[:-60]\chembelow{N}{H}(-[6,0.4,,,opacity=0])-NH_2}
                \arrow{->[$\Delta$][\ce{EtOH}]}
                \chemfig[fixed length=false]{*5([:-18]-N-=N-N(-R^1)=)}
            \schemestop
        \end{center}
        \item Acyl hydrazides and chloroimidates: Enables formation of same derivatives.
        \item \textbf{Pinner-type} (1,2,4-triazole synthesis): 13-substitution.
        \begin{center}
            \footnotesize
            \schemestart
                \chemfig{R^3-~N}
                \arrow{->[\tikz{\node[align=left]{1. \ce{MeOH}, \ce{HCl}\\2. \ce{R^1{}-NHNH2}\\3. \ce{HC(OEt)3}}}]}[,1.8]
                \chemfig[fixed length=false]{*5([:-18](-R^3)-\chembelow{N}{H}-=N(-R^1)-N=)}
            \schemestop
        \end{center}
    \end{itemize}
    \item Tetrazole synthesis.
    \begin{itemize}
        \item \textbf{Vilsmeier-Haack-type} (tetrazole synthesis): 15-substitution.
        \begin{center}
            \footnotesize
            \schemestart
                \chemfig{R^5-[:30](=[2]O)-[:-30]\chembelow{N}{H}-[:30]R^1}
                \arrow{->[\ce{NaN3}, \ce{Tf2O}][DCM, rt]}[,1.6]
                \chemfig[fixed length=false]{*5([:-18](-R^5)-N(-R^1)-N=N-N=)}
            \schemestop
        \end{center}
        \item Tin azide-type (tetrazole synthesis): 5-substitution.
        \begin{center}
            \footnotesize
            \schemestart
                \chemfig{R^5-~N}
                \+
                \chemfig{N_3-{}^{\emph{n}}Bu_3Sn}
                \arrow{->[\ce{HCl}]}
                \chemfig{*5([:-18](-R^5)-\chembelow{N}{H}(-[6,0.4,,,opacity=0])-N=N-N=)}
            \schemestop
        \end{center}
        \item \textbf{Passerini} (tetrazole synthesis): Enables formation of same derivatives.
    \end{itemize}
    \item Oxazole reactivity.
    \begin{itemize}
        \item S\textsubscript{N}Ar at C2 with good LG (e.g., chloride).
        \item C2-lithiation and electrophilic functionalization (via ring-opened isocyanide).
        \item 5-addition (2-lithiation, TIPS protection, 5-lithiation, functionalization, PG removal).
        \item Lateral deprotonation.
        \item \textbf{Cornforth rearrangement}.
        \begin{center}
            \footnotesize
            \schemestart
                \chemfig[fixed length=false]{R^2-*5(-O-(-R^5)=(-(=[::60]O)-[::-60]R^4)-N=)}
                \arrow{->[$\Delta$]}
                \chemfig[fixed length=false]{R^2-*5(-O-(-{\color{rex}R^4})=(-(=[::60]O)-[::-60]{\color{rex}R^5})-N=)}
            \schemestop
        \end{center}
    \end{itemize}
    \item Oxazole synthesis.
    \begin{itemize}
        \item \textbf{Robinson-Gabriel} (oxazole synthesis): 2(4)5-substitution.
        \begin{center}
            \vspace{0.5em}
            \footnotesize
            \schemestart
                \chemfig{R^5-[:30](=[:-30]O)-[2](-[:150]R^4)-[:30]\chemabove{N}{H}-[:-30](=[6]O)-[:30]R^2}
                \arrow{->[\ce{P2O5}][$\Delta$]}
                \chemfig[fixed length=false]{*5((-R^5)-O-(-R^2)=N-(-R^4)=)}
            \schemestop
        \end{center}
        \item \textbf{Bl\"{u}mlein-Lewy} (oxazole synthesis): 24-substitution.
        \begin{center}
            \footnotesize
            \schemestart
                \chemfig{R^4-[:-30](=[:30]O)-[6]-[:-150]Br}
                \arrow{0}[,0.1]\+
                \chemfig{R^2-[4](=[:-120]O)-[:120]H_2N}
                \arrow{->[$\Delta$]}
                \chemfig[fixed length=false]{*5(-O-(-R^2)=N-(-R^4)=)}
            \schemestop
        \end{center}
        \item \textbf{Fischer} (oxazole synthesis): Enables formation of same derivatives.
    \end{itemize}
    \item Isoxazole synthesis.
    \begin{itemize}
        \item Dipolar cycloaddition: 5-ester-34-substitution.
        \begin{center}
            \footnotesize
            \schemestart
                \chemfig{R^5-[2]~[2]-[2,,,3]MeO_2C}
                \arrow{0}[,0.1]\+{,,2.7em}
                \chemfig{HO-[:30]N=[2](-[:150]Cl)-[:30]R^3}
                \arrow
                \chemfig[fixed length=false]{*5((-R^5)-O-N=(-R^3)-(-MeO_2C)=)}
            \schemestop
        \end{center}
        \begin{itemize}
            \item Nitrile oxides formed from 1,3-elimination of oximes.
            \item Esters can be saponified.
            \item \ce{R^5} can be bromine, and then hydrogenated.
        \end{itemize}
    \end{itemize}
    \item Thiazole synthesis.
    \begin{itemize}
        \item \textbf{Hantzsch} (thiazole synthesis): 24(5)-substitution.
        \begin{center}
            \footnotesize
            \schemestart
                \chemfig{R^4-[:-30](=[:30]O)-[6](-[:-30]Br)-[:-150]R^5}
                \arrow{0}[,0.1]\+
                \chemfig{R^2-[4](=[:-120]S)-[:120]H_2N}
                \arrow
                \chemfig[fixed length=false]{*5((-R^5)-S-(-R^2)=N-(-R^4)=)}
            \schemestop
        \end{center}
        \begin{itemize}
            \item Other halides and pseudo-halides can be used in place of a bromide.
        \end{itemize}
        \item \textbf{Van Leusen} (thiazole synthesis): Enables formation of 4-tosyl-5-thioesterthiazoles.
        \item \textbf{Cook-Heilbron} (thiazole synthesis): Enables formation of 2-thio-5-amino-4-substituted thiazoles.
        \item \textbf{Robinson-Gabriel} (thiazole synthesis): 2(4)5-substitution.
        \begin{center}
            \vspace{0.5em}
            \footnotesize
            \schemestart
                \chemfig{R^5-[:30](=[:-30]O)-[2](-[:150]R^4)-[:30]\chemabove{N}{H}-[:-30](=[6]O)-[:30]R^2}
                \arrow{->[\ce{P4S10}][$\Delta$]}[,1.1]
                \chemfig[fixed length=false]{*5((-R^5)-S-(-R^2)=N-(-R^4)=)}
            \schemestop
        \end{center}
    \end{itemize}
    \item 1,2,4-oxadiazole synthesis.
    \begin{itemize}
        \item \textbf{Pinner-type} (1,2,4-oxadiazole synthesis): 35-substitution.
        \begin{center}
            \vspace{0.5em}
            \footnotesize
            \schemestart
                \chemfig{N~-R^3}
                \arrow{->[1. \ce{NH2OH*HCl}][2. \scriptsize\chemfig[atom sep=1.4em]{R^5-(=[:60]O)-[:-60]Cl}\hspace{2.3em}\ ]}[,2]
                \chemfig[fixed length=false]{*5([:-18](-R^5)-O-N=(-R^3)-N=)}
            \schemestop
        \end{center}
    \end{itemize}
    \item 1,3,4-oxadiazole sythesis.
    \begin{itemize}
        \item \textbf{Robinson-Gabriel-type} (1,3,4-oxadiazole syntheiss): 25-substitution.
        \begin{center}
            \vspace{0.5em}
            \footnotesize
            \schemestart
                \chemfig[fixed length=false]{R^5-(=[:-60]O)-[:60]OMe}
                \arrow{->[\ce{N2H4*$x$H2O}]}[,1.7]
                \chemfig[fixed length=false]{R^5-(=[:-60]O)-[:60]\chemabove{N}{H}-NH_2}
                \arrow{->[\scriptsize\chemfig[atom sep=1.4em]{X-[:-60](=[:-120]O)-R^2}]}[,1.2]
                \chemfig[fixed length=false]{R^5-(=[:-60]O)-[:60]\chemabove{N}{H}-\chemabove{N}{H}-[:-60](=[:-120]O)-R^2}
                \arrow{->[\ce{POCl3}]}
                \chemfig[fixed length=false]{*5([:-18](-R^5)-O-(-R^2)=N-N=)}
            \schemestop
        \end{center}
    \end{itemize}
\end{itemize}




\end{document}