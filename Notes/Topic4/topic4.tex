\documentclass[../notes.tex]{subfiles}

\pagestyle{main}
\renewcommand{\chaptermark}[1]{\markboth{\chaptername\ \thechapter\ (#1)}{}}
\setcounter{chapter}{3}

\begin{document}




\chapter{Organometallic Coupling Reactions}
\section{Organometallic Coupling Reactions}
\begin{itemize}
    \item \marginnote{2/27:}Today: Organometallic transformations that are bread and butter for pharmaceutical chemists, both in discovery and at scale.
    \item \textbf{Heck} (reaction).
    \begin{itemize}
        \item Many variants, but we'll focus on an aryl halide reacting with an olefin.
        \item Feature: We regenerate the double bond, as opposed to most couplings which increase saturation.
        \item Bio: Richard Heck.
        \begin{itemize}
            \item Started in industry at Hercules Corporation.
            \item Moved to University of Delaware.
            \item Was told that what he was doing wasn't interesting, so he quit, moved to Florida to raise orchids, and then moved to the Phillipines.
            \item A brilliant person who made contributions to a lot of fundamental mechanistic organometallic chemistry, as well. He was just ahead of his time, doing this stuff in the 60s-80s.
            \item Larry Overman, ??Tommy Oganachi??, etc. total synthesis people resurrected cross-coupling in academia and industry.
        \end{itemize}
        \item Basic mechanism.
        \begin{itemize}
            \item Oxidative addition.
            \begin{itemize}
                \item Forms a 16 electron, square-planar palladium species.
                \item Generally can't bind another ligand to go to an 18 electron species; that's high energy, so you dissociate a ligand.
            \end{itemize}
            \item Ligand exchange.
            \item Migratory insertion.
            \item $\beta$-hydride elimination.
            \begin{itemize}
                \item Very common, but can be constrained (Fu chemistry).
            \end{itemize}
            \item Ligand exchange.
            \item Reductive elimination.
            \begin{itemize}
                \item Running the reaction in the presence of a base drives the reaction by precipitation of the acid.
            \end{itemize}
        \end{itemize}
        \item Tri-\emph{o}-tolylphosphine was the ligand of choice for a while, because it has ?? that makes it dissociate more easily during ligand exchange.
        \item Small amounts of ?? can act as olefin isomerization catalysts and mess up reactions.
        \item Regioselectivity: Aryl group typically goes to less-substituted carbon, and metal typically goes to the more-substituted carbon.
        \begin{itemize}
            \item Rationalization: Steric factors and electronic factors.
            \item More electropositive palladium wants to go to the $\delta^-$ carbon.
            \item There is an added ionic component to \ce{Pd-C} bonds with certain EWGs/EDGs. Steve had to keep this ionic character in mind during his early research on early transition metal catalysis.
            \item Triflates can polarize palladium, and exaggerate this effect.
        \end{itemize}
    \end{itemize}
    \item Palladium-catalyzed carbon-nitrogen cross-coupling.
    \begin{itemize}
        \item Much more challenging to generalize than \ce{C-C} couplings.
        \begin{itemize}
            \item With basic, nitrogen compounds, you have compounds that were previously used as ligands and compete for open coordination sites.
            \item The balance is keeping palladium in solution ("you fear the precipitation of the dreaded palladium black") with ligands that don't let go.
        \end{itemize}
        \item Aryl halides and anilines are common.
        \item Reagents.
        \begin{itemize}
            \item \ce{Pd(OAc)2} is relatively cheap, but it needs to be reduced before the chemistry starts.
            \item \ce{Pd2(dba)3} is slightly more expensive, in the right oxidation state, but dba is hard to get rid of.
        \end{itemize}
        \item A history of ligands.
        \begin{itemize}
            \item Instead of amines, use amido-stannanes. Tri-\emph{o}-tolylphosphine ligands make this work (Migita, Kosugi). Amido-stannanes are terrible to work with, though.
            \item Then the chemistry went to bidentate phosphines, then back to monodentate phosphines, then NHCs.
            \item Most widely used ligands: Xantphos and racemic BINAP.
            \item BippyPhos was developed to get around patents that MIT held; ironically developed by one of Steve's former postdocs.
        \end{itemize}
        \item Proposed catalytic cycle.
        \begin{itemize}
            \item Particularly for \ce{C-N} coupling, what's really going on is very messy. You want to keep stuff on-cycle, but there's all sorts of off-cycle equilibria.
            \item Oxidative addition.
            \begin{itemize}
                \item Used to be rate- and yield-determining, but no longer kinetically relevant.
                \item Thus, it's better to not use aryl iodides now. Iodides are more expensive, their waste disposal is more expensive, and halogen loss is slower with sterically huge iodine.
            \end{itemize}
            \item There exists a sensitivity to aliphatic amines vs. anilines.
        \end{itemize}
        \item BINAP.
        \begin{itemize}
            \item Racemic BINAP is very cheap. BINAP was developed as a ligand for asymmetric hydrogenation by Noyori.
            \item Racemates typically have a ??higher?? melting point than individual enantiomers (because of \textbf{eutectic mixtures}; recall from PChem).
            \item Triarylphosphine: Good electron donor, but not a fantastic one. Thus, very good for aryl bromides and triflates (which have relatively easy oxidative addition); not good for iodides due to the formation (presumably) of bridging compounds.
            \item Many solvents good.
            \item Strong bases and weak bases both good.
        \end{itemize}
    \end{itemize}
    \item Example synthesis: KRAS inhibitor.
    \begin{itemize}
        \item Up to \SI{300}{\kilo\gram} scale with BINAP!
    \end{itemize}
    \item Xantphos.
    \begin{itemize}
        \item Invented by Pete Van Leeuwen when he was at Dutch Shell for hydroformylation (how all linear and branched alcohols are prepared, as well as butyraldehyde).
        \item Billions of dollars were spent trying to change the ratio of linear to branched butyraldehyde, and this came out of that.
        \item It's a good surrogate for BINAP in many reactions.
        \item Only works with very activated heteroaryl chlorides.
        \begin{itemize}
            \item Example: 2-chloropyridine is an honorary aryl bromide.
        \end{itemize}
        \item The slides list a good (albeit now a bit dated) review of the prior 10 years of cross-coupling.
    \end{itemize}
    \item Tri-\emph{t}-butylphosphine.
    \begin{itemize}
        \item Used in many coupling reactions of heterocycles.
    \end{itemize}
    \item Bulky mono-phosphines.
    \begin{itemize}
        \item Air-stable.
        \item Tons have been prepared and legally sold; "more tons have probably been prepared and\dots not legally sold."
        \item Steve reviews the benefits of tetrakis vs. single-coordinate debate.
        \begin{itemize}
            \item As the cone angle increases, the amount of \ce{L1Pd} increases.
        \end{itemize}
        \item It's only the interaction of the \emph{ipso}-carbon (bound to upper ring) with the palladium that matters, not the whole bottom ring as is often incorrectly drawn.
        \item At some point, the ligand gets too big and you reach an unstable situation.
    \end{itemize}
    \item How do you form \ce{Pd^0}?
    \begin{itemize}
        \item It doesn't matter how active your catalyst is if you never form it!
        \begin{itemize}
            \item Steve has often told his students to confirm that their catalyst is being formed if a reaction isn't working.
        \end{itemize}
        \item Out-of-the-bottle \ce{Pd^0} complexes come with extra-ligand baggage.
        \item Kinetic studies by ?? have really shown that extra dba slows reations.
    \end{itemize}
    \item Solution: Mechanism-based activation.
    \begin{itemize}
        \item Put the middle of your catalytic cycle into your pre-catalyst! Then you get deprotonation, reductive elimination, etc.
        \item Biscoe developed the first one, and it worked. Could make it on a \SI{100}{\gram} scale. But if you put it in solution, it would decompose.
        \item Yong could do multi-kilo synthesis, very simple preparation.
        \item Carbazoles aren't cool in Europe (environmental concerns).
    \end{itemize}
    \item Coupling of anilines and aryl chlorides.
    \begin{itemize}
        \item Papers often get into JACS or \emph{Science} with really active catalysts (0.01-0.05 mol\%), but in Steve's opinion, there's no point to these catalysts if nobody wants any of the compounds they can be used to produce (i.e., if substrate scope is too small).
        \item The vast majority of synthetic methods aren't useful in any real circumstances. What matters is if you can do the chemistry on complicated substrates.
        \item The vast majority of people practicing the chemistry are in discovery chemistry, so you should target your work to them.
    \end{itemize}
    \item Example synthesis: Gleevec.
    \begin{itemize}
        \item This is great, even though you've got a free \ce{NH} and tons of different nitrogens.
        \item Common issue: Substrates and products can have poor solubility.
    \end{itemize}
    \item Example synthesis: Amgen compound.
    \begin{itemize}
        \item Optimized catalytic conditions.
        \item Functionalized silica gel with thiourea stuff helps get rid of the palladium.
    \end{itemize}
    \item Wacker oxidation.
    \begin{itemize}
        \item Commerically makes acetaldehyde from ethylene.
        \item Amazingly efficient: Low price difference between acetaldehyde and ethylene so it \emph{has} to be super efficient.
        \item Palladium, copper, air, and catalytic acid.
        \item The palladium in this reaction \emph{loves} terminal olefins.
        \item You form a cationic \ce{Pd^{II}} complex that binds the olefin. Water adds, enolization to the ketone.
    \end{itemize}
    \item Lou Hegedus's chemistry.
    \begin{itemize}
        \item Like Heck, he was too far ahead of his time for his own good. Avid fisherman. If he had invented it 20 years later, he would have been a superstar, but at the time, nobody thought it could be used.
        \item This is ring-closing Wacker oxidation!!
        \item Can be used for indole synthesis.
        \item $\pi$-allyl (Tsuji-Trost) chemistry for the bottom left step.
        \item Uses palladium for every step in this synthetic scheme! Like a competition to see how much palladium you can do.
    \end{itemize}
    \item \textbf{Cacchi} (indole synthesis).
    \begin{itemize}
        \item \emph{ortho}-alkynyl aniline, with a protected N.
        \item Net transformation: \emph{trans}-addition of a nitrogen and an aryl group across an alkyne.
        \item General principle: If you can do it once, it's good; if you can do it twice, it's better.
        \begin{itemize}
            \item Thus, it's great that you can do it at two sites in the bottom example!
        \end{itemize}
    \end{itemize}
    \item \textbf{Larock} (indole synthesis).
    \begin{itemize}
        \item Larock (now retired from ISU, interesting chemistry in the 70s).
        \item Quite wide scope; can now be done with bromides and chlorides.
        \item You essentialy annulate on the rest of the indole.
    \end{itemize}
    \item \textbf{Mori-Ban} (indole synthesis).
    \begin{itemize}
        \item Heck-type palladium coupling.
        \item Used by Jim Cook to make substituted tryptophan derivatives.
        \item \textbf{Sch\"{o}llkopf's reagent} is an anionic amino acid equivalent.
    \end{itemize}
    \item \textbf{Merck} (indole synthesis).
    \begin{itemize}
        \item Highlights the limitations of the Larock indole synthesis.
        \item DABCO is the ligand; a very common base used in pharmaceutical chemistry.
        \item Condense to the enamine, oxidative addition, attack at \ce{Pd^{II}}, then reductive elimination and aromatization.
    \end{itemize}
    \item More on the Fischer indole synthesis.
    \begin{itemize}
        \item Limitation: Requires aryl hydrazines.
        \begin{itemize}
            \item Potent skin sensitizers, and have a multistep synthesis.
        \end{itemize}
        \item So\dots
        \begin{itemize}
            \item Almost any palladium catalyst will form the desired aryl hydrazine \emph{in situ}, and then we can do the Fischer indole synthesis.
            \item This is the \textbf{Buchwald modification} (of the Fischer indole synthesis).
        \end{itemize}
    \end{itemize}
    \item Example: Non-nucleotide reverse transcriptase inhibitor.
    \begin{itemize}
        \item Can do a second functionalization with the Fischer indole variant.
    \end{itemize}
    \item Cu-catalyzed \ce{C-N} bond formation.
    \begin{itemize}
        \item History.
        \begin{itemize}
            \item Started much earlier than palladium chemistry.
            \item This is Ullmann and Goldberg chemistry.
            \item Problem: They didn't have much mechanistic understanding, so they thought ligands were bad for the reaction.
            \item Stoichiometric strong base and very polar solvents meant that high temperatures were required.
            \item So the chemistry worked in some cases and not in others.
            \item But in the 1990s, this chemistry was brought back to the fore and ligands were developed.
        \end{itemize}
        \item Aside/maxim: The most expensive thing you have in discovery chemistry is time, so you just want stuff to work as rapidly as possible.
        \item Many ligands good.
        \begin{itemize}
            \item Very different selectivities.
        \end{itemize}
        \item Amides is the \textbf{Irma Goldberg coupling}.
        \item Ullmann discovered the original chemistry; Ullmann and Goldberg were married!
        \item You want the ligands to be good enough that multiple nitrogen species won't bind.
        \item Many different proposed mechanisms.
        \begin{itemize}
            \item Oxidative addition/reductive elimination has the most support so far.
            \item Caveat: Sensitivity of the reaction to the electronic nature of the aryl halide (think $\rho$ and Hammett plots). For palladium-catalyzed oxidative addition, $\rho\approx 3.5$, so it's quite sensitive to the electronic environment. But in copper chemistry, $\rho\approx 0.3-0.5$.
            \item Additionally, copper is much more sterically hindered at the substrates.
        \end{itemize}
        \item Ma's oxalamide ligands; Steve agrees with his interpretation.
        \item Copper doesn't tend to work for aryl triflates. It probably is some kind of coupled electron transfer.
        \item Primary amides and small-ring $\beta$-lactams are good to use.
        \item People say that sulfur and nitrogen poison palladium, but there are exceptions.
        \item Goldberg reaction.
        \begin{itemize}
            \item Irma Goldberg broke the glass ceiling because her reactions were just that important.
        \end{itemize}
    \end{itemize}
    \item Applications with diamine ligands.
    \begin{itemize}
        \item Doing the chemistry in the presence of added iodide does the copper chemistry more efficiently.
    \end{itemize}
\end{itemize}




\end{document}