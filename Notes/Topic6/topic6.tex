\documentclass[../notes.tex]{subfiles}

\pagestyle{main}
\renewcommand{\chaptermark}[1]{\markboth{\chaptername\ \thechapter\ (#1)}{}}
\setcounter{chapter}{5}

\begin{document}




\chapter{Azoles}
\section{Triazoles, Tetrazoles, Oxazoles, Thiazoles, and Combinations}
\begin{itemize}
    \item \marginnote{3/6:}Announcements.
    \begin{itemize}
        \item Exam 1.
        \begin{itemize}
            \item Spread: 57-96.
            \item Median: 90.888
            \item Mean: 88.
        \end{itemize}
        \item Exam 2 will be similar but cover some content that wasn't on Exam 1.
        \item The presentations should be submitted to Dennis by Monday at noon; if that's a problem, let them know ASAP.
        \begin{itemize}
            \item Let them know if we're not going to be here for any dates.
            \item Order: The numbering of presentations. 1-7, 8-14, 15-21.
            \item People tend to say the presentation is one of the things they get the most benefit out of in the course.
            \item If we use a Mac, send as a PPTX or PDF file; PC, send as a PDF.
        \end{itemize}
        \item Today's slides are freshly edited!! Make sure you have the right copy from Canvas.
    \end{itemize}
    \item We now begin lecture.
    \item Applications with diamine ligands.
    \begin{itemize}
        \item Having an iodide source in solution cross-couples \ce{KI} to the aryl bromide, forming an aryl iodide intermediate first.
    \end{itemize}
    \item Cu-catalyzed coupling of $\alpha$-amino acids.
    \begin{itemize}
        \item Pioneer: Dawei Ma, Shanghai University.
        \begin{itemize}
            \item Originally used palladium and copper, but "a sage reviewer (who was me)" told him that he probably didn't need the palladium.
        \end{itemize}
        \item Copper binds to carboxylate and does an intermolecular transfer.
        \begin{itemize}
            \item Copper is good if you have \emph{ortho}-carboxylates on an aryl chloride, even back in Ullmann's time.
            \item Essentially, the $\alpha$-carboxylate on the amine substrates helps stabilize the intermediate, like a ligand!
        \end{itemize}
        \item Indeed, sometimes you also don't need any added ligand!
        \begin{itemize}
            \item These are very efficient reactions because copper ligands are usually more expensive than the copper itself!
        \end{itemize}
        \item Copper oxide (\ce{Cu2O}) is also the cheapest form of copper to use at scale, if you're able to use it.
        \begin{itemize}
            \item \ce{Cu^{0/I/II}} are all chemically competent in this reaction and interconvert.
        \end{itemize}
    \end{itemize}
    \item Cu-catalyzed \ce{C-N} reactions with amines.
    \begin{itemize}
        \item Dawei Ma's work again.
        \item Originally used dimethylglycine or proline.
        \begin{itemize}
            \item Moved on to inventing doubly deprotonated, anionic oxalamide ligands. These put more electron density onto the copper center than when you do the reactions with neutral diamine ligands. The difference between an anion and lone pair donation.
        \end{itemize}
        \item The interesting ones from a pharmaceutical perspective are the ones that tolerate heterocycles on both sides.
        \item Quinazoline synthesis.
        \begin{itemize}
            \item First step is Goldberg-type coupling, then condensation.
        \end{itemize}
    \end{itemize}
    \item Significant breakthrough: The Ma paper on Cu-catalyzed amination of (hetero)aryl chlorides.
    \begin{itemize}
        \item This is still not ultra-user friendly (\SI{120}{\celsius}), but it does have relatively broad utility and scope.
        \item Oxalamides are quite modular to put together: Amine plus oxalyl chloride. Thus, easy to do structure-activity relationships/high-throughput!
    \end{itemize}
    \item Moving away from metal-catalyzed couplings.
    \item \textbf{Triazole}: A five-membered heterocycle with three nitrogens.
    \begin{itemize}
        \item Lots of triazoles are fungicides.
        \item Fungal infections in the hospital are a serious source of death, so there is a big need for such oral antifungals.
    \end{itemize}
    \item \textbf{Contiguous} (triazoles): 1,2,3-triazoles.
    \begin{itemize}
        \item There are also \textbf{1,2,4-triazoles}.
    \end{itemize}
    \item Triazoles are actually super stable, despite all the nitrogens; can heat them up to \SI{500}{\celsius}.
    \begin{itemize}
        \item The click reaction is now omnipresent in biochemistry.
    \end{itemize}
    \item Triazole chemistry.
    \begin{itemize}
        \item Rapid tautomerization.
        \item Amphoteric, like with imidazole. Protonated form has $\pKa=1.2$.
        \begin{itemize}
            \item Could I click together a DOT derivative??
        \end{itemize}
    \end{itemize}
    \item Synthesis of triazoles.
    \begin{itemize}
        \item $[3+2]$ disconnections, predominantly.
        \begin{itemize}
            \item It's important to remember the original progenitors of the work.
            \item Huisgen did the original work; it was in the literature for 30-40 years before Sharpless figured out that copper could seriously accelerate the rate.
        \end{itemize}
        \item Aliphatic azides: Do S\textsubscript{N}2.
        \item Aryl azides: Go through diazoniums.
    \end{itemize}
    \item Triazoles for click chemistry.
    \begin{itemize}
        \item Sharpless really got people thinking about click chemistry (though everyone thought he was crazy when he first pitched it).
        \begin{itemize}
            \item They have parking spots at Scripps for Nobel laureates, and now one for people with two (so Sharpless has his own).
            \item Steve: "I hope that Barry doesn't drive, but ok."
        \end{itemize}
        \item Bertozzi is the one who realized the chemistry.
        \item Copper catalytic cycle.
        \begin{itemize}
            \item Make copper acetylide (as in Sonogashira chemistry), then coordinate the azide. Form the allene (maybe a diradical), ring contraction, then elimination.
            \item This is relatively believed, but it could well be fantasy per Steve.
        \end{itemize}
        \item Ruthenium version.
        \begin{itemize}
            \item From Fokin. Originally worked with Barry was critical to the chemistry, but they had a falling out and he moved to USC.
            \item In this version, you get the 1,5-isomer. These are regio-complimentary processes.
            \item Very good functional group compatibility.
            \item Different mechanism, even from an early organometallic chemistry. Bind alkyne, displace a ligand with azide, cycloaddition to a compound that reductively eliminates to do \ce{C-N} bond formation, and then falls off. Mechanistic hypothesis by Pierre Dixneuf.
        \end{itemize}
    \end{itemize}
    \item 1,2,4-triazole synthesis.
    \begin{itemize}
        \item Paal-Knorr-type mechanism.
        \begin{itemize}
            \item Cyclodehydration using \ce{P4O10} (a non-HCl-producing version of \ce{POCl3}).
            \item Method is not mild: \SI{250}{\celsius}.
        \end{itemize}
        \item \emph{Sym}-triazine (etymology is unknown to Steve) plus primary hydrazine makes a 1,2,4-triazole.
        \begin{itemize}
            \item Mechanistically interesting: Carbons are rarely activated for attack. Then electrocyclic ring-opening. Then intramolecular attack to form a species that ejects the simple amidide.
        \end{itemize}
        \item Acyl hydrazides and chloroimidates.
        \begin{itemize}
            \item Heating in the presence of base leads to cyclization, after activation.
        \end{itemize}
        \item Pinner strategy.
        \begin{itemize}
            \item Nitrile to a Pinner salt, condense with hydrazine, and add one carbon with an orthoester.
            \item Know how to write the mechanism for orthoester stuff (could be on Exam 2)!!
        \end{itemize}
    \end{itemize}
    \item Example synthesis: 1,2,4-triazole.
    \begin{itemize}
        \item S\textsubscript{N}Ar synthesis of triazoles.
        \begin{itemize}
            \item Gives a 3:2 mixture of different isomers. Selectivity from sterics of the methyl group.
            \item Then S\textsubscript{N}Ar with \ce{MeO-} and removing the methyl group with \ce{BBr3}.
        \end{itemize}
        \item Other strategy: Multicomponent synthesis.
        \begin{itemize}
            \item Nitrogen-nitrogen bond formation is an unmet need in organic chemistry!
            \item Here you do displacement with a nitrogen nucleophile and nitrogen electrophile??
        \end{itemize}
    \end{itemize}
    \item We'll not discuss pentazoles, but\dots
    \item Tetrazoles.
    \begin{itemize}
        \item Tetrazoles are pharmaceuticals substitutes for carboxylic acids!
        \begin{itemize}
            \item Very acidic.
            \item But slower pharmacokinetics; the body excretes carboxylic acids very fast.
        \end{itemize}
        \item Rapid tautomerization.
    \end{itemize}
    \item Making tetrazoles without azides is very difficult.
    \begin{itemize}
        \item Can do with triflic anhydride to form an activated intermediate that can do addition-elimination, then intramolecular displacement of nitrogen onto the end.
        \begin{itemize}
            \item Note: Using an azide with DCM forms a small amount of 1,1-diazidomethane, which is a contact explosive and can blow up and cause a secondary explosive from the azide. So don't repeat this procedure!
        \end{itemize}
        \item Stannyl tetrazole.
        \begin{itemize}
            \item This is one exception where alkyltin reagents can be used on scale because higher molecular weight azides are safer than low molecular weight azides, even though we don't typically like tin chemistry at scale.
            \item Maybe do my DOT synthesis with tributyltin azide instead of \ce{HN3}??
        \end{itemize}
        \item Classic methods, rejuvenated during the heyday of combinatorial chemistry.
        \item \textbf{Passerini} (tetrazole synthesis).
        \begin{itemize}
            \item Don't do in general because \ce{HN3} is a very explosive compound. In small amounts in flow, it's probably fine, but not good in big amounts.
            \item Proposed mechanism: Nitrilium??
        \end{itemize}
    \end{itemize}
    \item Example synthesis: Biggest selling tetrazole of all time (Novartis chemical; billions of dollars per year): Sartans, esp. valsartan. Huge for high blood pressure, heart failure, and kidney disease caused by diabetes.
    \begin{itemize}
        \item Benzylic bromination radical reaction.
        \item Displace with valine (protected as benzyl ester).
        \item \textbf{Schotten-Baumann} (amide synthesis): The classic way to make amides. Interfacial reaction: Amide bond formation is faster than hydrolysis of the acid chloride. Amine is more soluble in the organic layer than the water is. Very efficient: Shake two things together, then filter off the organic layer and you have all the product!
        \item Heated with tributylstannyl azide in xylene to form the triazole, then reduce the ester.
    \end{itemize}
    \item Example synthesis: 6 nitrogens and 7 carbons.
    \begin{itemize}
        \item Claisen condensation to diketone, then condense with hydrazine to make the pyrazole carboxylic ester.
        \item Hydrolyze under standard conditions. Treat with CDI (safer phosgene), then ammonia.
        \item Amide dehydration (this time with TFAA).
        \item \emph{In situ}-generated zinc azide does the cyclization to the tetrazole.
    \end{itemize}
    \item Example synthesis: Key step related to an Exam 1 problem.
    \begin{itemize}
        \item Amino-chlorodipyrimidine. Think about how to make this!!
        \item \emph{para}-methoxybenzylchloride made fresh, because very reactive and produces \ce{HCl}.
        \item Kumada coupling to vinyl pyrimidine acts like an $\alpha,\beta$-unsaturated carbonyl. Can do nucleophilic attack.
        \item Iodinated under acidic conditions.
        \item Copper-catalyzed \ce{C-C} coupling. Hydrolysis/decarboxylation.
        \item TMS-azide with triethyl orthoformate (alternative to \ce{Bu3SnN3}??)
    \end{itemize}
    \item Oxazoles.
    \begin{itemize}
        \item \textbf{Kemp elimination} (named after MIT's Dan Kemp): Ring-opening under strongly basic conditions.
        \item Nitrogen is mildly basic; poor resonance.
        \item Very common in pharmaceuticals.
        \item Aromaticity: Nitrogen is pyridine-like, one oxygen lone pair in the aromatic system ($sp^2$-hybridized oxygen).
        \item Relatively acidic at the 2-position. Can be deprotonated with standard bases (e.g., LDA).
        \item Generally do not undergo EAS.
    \end{itemize}
    \item Oxazole reactions.
    \begin{itemize}
        \item Can get S\textsubscript{N}Ar if good leaving group.
        \item C-metallations: 2-lithio compound is in equilibrium with ring-opened enolate isocyanide; can be trapped as \ce{O-TMS} ether. If you heat that, it (remarkably) rearranges to what you thought you were gonna make in the first place!
        \begin{itemize}
            \item Exclusive C-silylation: More hindered, but triflates react more rapidly. Perhaps this reaction is done reliably cold, instead of while heating, wonders Steve?
        \end{itemize}
        \item Lateral deprotonation.
    \end{itemize}
    \item John Cornforth: Another big chemist who helped invent chemical biology.
    \begin{itemize}
        \item Won the Nobel prize with Prelog.
        \item Incredibly talented (and deaf!).
        \item One reaction (not his NP one) was the \textbf{Cornforth rearrangement}.
        \begin{itemize}
            \item Occurs via ring-opening to pseudo-symmetric intermediate, from which you can close back equally well.
            \item \ce{CH3} vs. \ce{CD3} \ce{R}-groups would give a 50/50 mixture.
        \end{itemize}
    \end{itemize}
    \item Synthesis.
    \begin{itemize}
        \item The usual suspects.
        \item \textbf{Robinson-Gabriel} (oxazole synthesis): $\alpha$-amino ketone, acylate it, and dehydrogenate with \ce{P2O5} (which, remember, is really \ce{P4O10}).
        \begin{itemize}
            \item Acid-promoted cyclization and loss of water.
            \item Driving force: Making phosphoric acid. \ce{P2O5} is like a phosphorous anhydride, so you're forming stronger \ce{P=O} bonds in this reaction.
        \end{itemize}
        \item \textbf{Bl\"{u}mlein-Lewy} (oxazole synthesis).
        \begin{itemize}
            \item Steve goes over mechanism.
            \item Can also push arrows from the nitrogen and lose water afterwards.
        \end{itemize}
        \item \textbf{Fischer} (oxazole synthesis): Condensation of a cyanohydrin with an aldehyde under \ce{HCl} conditions.
        \begin{itemize}
            \item Cyanohydrins in basic conditions release cyanide, so keep it acidic!
        \end{itemize}
        \item \textbf{Van Leusen} (oxazole synthesis): Skipping mostly. \emph{Also known as} \textbf{Sch\"{o}lkopf}.
    \end{itemize}
    \item Isoxazole synthesis.
    \begin{itemize}
        \item \ce{N-O} bonds makes these difficult to work with in a lot of cases.
        \item Synthesis by dipolar cycloaddition.
        \begin{itemize}
            \item Can make nitrile oxides from 1,3-elimination or dehydration of nitro compounds.
            \item Dipolar cycloaddition on enamine example: May be E2; may be anomeric effect from oxygen kicking out the amine, followed by loss of the proton.
        \end{itemize}
    \end{itemize}
    \item Thiazoles.
    \begin{itemize}
        \item More basic than oxazoles (because sulfur is less electronegative).
        \item Very common in drugs and dyes.
        \item More aromatic than oxazoles (again, sulfur is less electronegative).
        \item Thiazoliums are precurors to organocatalysts, e.g., thiamine pyrophosphate
        \item Analogous electronic structure (pyridine nitrogen, $sp^2$ sulfur) to oxazole.
    \end{itemize}
    \item Thiazole reactivity: EAS has preferential sites, even though it's not very common.
    \item Thiazole syntheses.
    \begin{itemize}
        \item The usual suspects: Hantzsch, Van Leusen, Robinson-Gabriel.
        \item \textbf{Hantzsch} (thiazole synthesis): Thioamide. Electrophile attacks at the sulfur because it's polarizable and nucleophilic.
        \begin{itemize}
            \item Then hemiaminal formation, followed by the splitting out of water.
            \item Used in the total synthesis of Pomothlocin A.
        \end{itemize}
        \item \textbf{Van Leusen} (thiazole synthesis): Always start with TosMIC, deprotonation, addition to an electrophile, and then cyclization back onto the isonitrile carbon. Can acylate the \ce{S-}, though it probably wouldn't survive workup.
        \item \textbf{Cook-Heilbron} (thiazole synthesis): Many thiazole syntheses use \ce{CS2}. \ce{CS2} used to be a common solvent, and it smells horrible.
        \begin{itemize}
            \item Again, \ce{CS2} looks like \ce{CO2} in terms of reactivity: Attack at the central carbon.
            \item Sulfur adds to pendant nitrile, then tautomerization to substituted thiazole.
            \item Variation with $\alpha$-haloketones as well.
        \end{itemize}
        \item \textbf{Robinson-Gabriel} (thiazole synthesis): Use \ce{P4S10}, or Lawesson's reagent for a gentler method. Then cyclization with loss of water.
    \end{itemize}
    \item Example synthesis: Sodelglitazar.
    \begin{itemize}
        \item Get to thioamide, treat with $\alpha$-chloro-$\beta$-ketoester.
        \item Very good, old-fashioned chemistry.
    \end{itemize}
    \item Oxadiazoles.
    \begin{itemize}
        \item Purported to be more important in the future.
        \item Showing up more and more, e.g., in a new explosive.
    \end{itemize}
    \item 1,3,4-oxadiazole synthesis.
    \begin{itemize}
        \item Hydrazine solution and hydrazine hydrate are more stable than anhydrous hydrazine.
        \item Substitute to form hydrazide, trap with something else, then dehydrate (Robinson-Gabriel type).
    \end{itemize}
    \item 1,2,4-oxadiazole synthesis.
    \begin{itemize}
        \item Pinner-type chemistry.
    \end{itemize}
    \item Example synthesis: PPAR-a agonist.
    \begin{itemize}
        \item Covers this.
    \end{itemize}
    \item Example synthesis: Apelin receptor (APJ) agonist.
    \begin{itemize}
        \item Ethoxylated ethyl acetoacetate, convert to vinyligous urethane, treat with diethyl malonate to make hydroxy-pyridone. Halogenate.
        \item Negishi coupling.
        \item Treat with hydrazide to form an intermediate that they can then dehydrate with T3P.
    \end{itemize}
\end{itemize}




\end{document}