\documentclass[../notes.tex]{subfiles}

\pagestyle{main}
\renewcommand{\chaptermark}[1]{\markboth{\chaptername\ \thechapter\ (#1)}{}}
\setcounter{chapter}{2}

\begin{document}




\chapter{\texorpdfstring{$\bm{\pi}$}{TEXT}-Excessive Heterocycles}
\section{Imidazoles, Pyrazoles, and Indoles}
\begin{itemize}
    \item \marginnote{2/20:}Announcements.
    \begin{itemize}
        \item PSet 2 will not be posted yet because most of the material won't be covered until next Tuesday.
        \begin{itemize}
            \item We'll still have it a week before the exam, and the exam will not be so indole focused.
        \end{itemize}
        \item The practice exams are also still to come.
        \item Lots of material and esoteric reactions in these slides; Steve will not discuss it all, nor expect that we remember it all.
    \end{itemize}
    \item \textbf{Imidazole} and \textbf{benzimidazole}.
    \begin{itemize}
        \item Important constituents in pharmaceuticals and biologically important substructures (e.g., histidine; nucleophile in salt bridges; constituent in DNA).
        \item Proteins are often purified on nickel columns that act on histidines (see "HisTags").
    \end{itemize}
    \item Structure and reactivity.
    \begin{itemize}
        \item Often put into structures to increase water solubility (can completely hydrogen-bond; both donor and acceptor)!
        \item Combination of pyridine and pyrrole: One lone pair orthogonal to the $\pi$-system, and one pyrrole-like pair that does not typically react with electrophiles.
        \item Imidazole is less nucleophilic than pyrrole at carbon.
        \item Rapid tautomerization complicates reactivity; if you want to target a particular site, you might get a surprise. But there are ways to overcome this that we'll discuss.
        \item Imidazole is amphoteric: One \ce{H} is moderately acidic (not super, but not like \ce{C-H} either), and then can protonate. Much less acidic than oxazole or thiazole because of resonance.
        \item Alkylation at nitrogen occurs, followed by deprotonation, followed by more reactivivity.
    \end{itemize}
    \item Reactions of imidazole.
    \begin{itemize}
        \item Deprotonation (with a strong base, e.g., \ce{NaH}, LDA, \ce{NaHMDS}) creates a strong base that monoalkylates.
        \item Selective alkylation at nitrogen?
        \begin{itemize}
            \item Target \ce{N^2} with protection (acylation), \textbf{ethyl Meerwein's reagent}, and deacylation.
            \item Target \ce{N^1} with Buchwald amination (Steve: "I hate the amination, should've gotten rid of the amination").
        \end{itemize}
        \item EAS.
        \begin{itemize}
            \item Better than benzene, worse than pyrrole.
            \item Nitration breaks the symmetry of the molecule. Easier to put nitrogen group next to non-positively charged nitrogen.
            \item Polybromination is also possible.
            \item Selective bromination occurs analogously to with pyridine (see Figure \ref{fig:TTQPy3Br}); attack at nitrogen, then carbon, then rearomatization.
            \item More "trivial" reactions.
        \end{itemize}
        \item S\textsubscript{N}Ar.
        \begin{itemize}
            \item Moves the lone pair onto the nitrogen, as we've seen.
        \end{itemize}
        \item Directed metallation.
        \begin{itemize}
            \item SEM (popularized by Bruce Lipschutz at UC-Santa Barbara) is the best protecting group. Can be removed by fluoride, which induces a loss of ethylene and formaldehyde.
            \item Selective deprotonation between the two nitrogens (fairly standard, steric factors considered).
            \item Can then do again.
        \end{itemize}
        \item Lithium-halogen exchange.
        \begin{itemize}
            \item LiX exchange occurs faster than deprotonation, then deprotonation occurs.
            \item To ensure that everything occurs in the right order, people will often add a strong base (e.g., \ce{LiHMDS}) first; then add butyl lithium to do the LiX exchange.
        \end{itemize}
        \item Radical chemistry, e.g., the Minisci reaction.
        \begin{itemize}
            \item Photochemistry as well, but that would be a whole other course; Steve won't discuss, take 5.44 with Alison if you want to hear more.
            \item Minisci (radical decarboxylation) predates photoredox catalysis for generation of carbon-centered radicals.
            \item Second example: Nucleophilic radical can add to electron-deficient (because of the aldehyde) carbon center.
        \end{itemize}
        \item Quaternary imidazolium salts.
        \begin{itemize}
            \item Subsequent base yields ylide, i.e., the NHC (NHCs ubiquitous in catalysis).
        \end{itemize}
    \end{itemize}
    \item Selected imidazole disconnections.
    \begin{itemize}
        \item Some should look familiar, and some may not.
        \item The first one to talk about individually is the \textbf{Debus-Radziszewski} (imidazole synthesis).
        \begin{itemize}
            \item From a long, long time ago. First reported synthesis of imidazole.
            \item Combines a 1,2-diketone, aldehyde, and ammonia.
            \item Proposed mechanism has zero evidence, but some variation is probably correct.
            \begin{itemize}
                \item Aldehyde is probably converted to imine \emph{before} formation of the diimine on the 1,2-diketone.
                \item Then condensation.
                \item Then tautomerization.
            \end{itemize}
        \end{itemize}
        \item Synthesis 1: Analogous to the Pinner reaction; very common.
        \item Van Leusen: Analogous to the pyrrole synthesis of the same name.
        \item Synthesis 4.
        \begin{itemize}
            \item Aminoacetal an acetal for stability reasons.
            \item Attack to imine and then cyclization.
        \end{itemize}
        \item Synthesis 6: Paal-Knorr type.
    \end{itemize}
    \item Example synthesis: Conivaptan.
    \begin{itemize}
        \item Pinner-type synthesis.
    \end{itemize}
    \item Example synthesis: Estrogen receptor.
    \begin{itemize}
        \item \ce{N-} adds to nitrile.
        \item Workup to amidine.
        \item Condense with $\alpha$-bromoacetaldehyde to form the imidazole.
    \end{itemize}
    \item Example synthesis: Obesity.
    \begin{itemize}
        \item $\alpha$-bromopyruvate.
    \end{itemize}
    \item Van Leusen.
    \begin{itemize}
        \item TosMIC: Stabilized isocyanide.
        \item Easily deprotonated, add to the imine, attack at carbene, proton transfer, losing the sulfonate.
    \end{itemize}
    \item COX-2 inhibitors.
    \begin{itemize}
        \item Historically important chemistry.
        \begin{itemize}
            \item Merck billion-dollar molecule.
            \item Has to do with pain.
            \item Aspirin (but disrupts stomach) $\to$ NSAIDs (ibuprofen, endoproxin) $\to$ tylenol (but dissolves liver) $\to$ opioids (but addictive).
            \item Most things inhibit both pathways (COX-1 and COX-2), but this drug was selective for COX-2, specifically. But this (Vioxx) causes heart-valve problems (and Merck had to pull it from the market at great loss).
            \item Celebrex as well, but the company died and had to be sold to Pfizer.
        \end{itemize}
    \end{itemize}
    \item Example synthesis: Like Lipitor, another statin compound.
    \begin{itemize}
        \item Glycine benzyl ester is a fairly standard protected amino acid.
        \item Treat it to form something, which after cleavage can be acylated.
        \item Ester to benzylamine.
        \item Cyclize with a primary amine to stitch in the nitrogen.
        \item Cyclize (fairly typical with statins).
    \end{itemize}
    \item Example synthesis: Debus-Radziszewski chemistry.
    \begin{itemize}
        \item Microwave chemistry was huge, but the bubble has burst at this point. You still see it here and there, but not much.
        \item Get to the asymmetric $\alpha$-diketone with a \ce{SeO2} oxidation.
        \item What method you'll use commonly depends on what you have and what you have successfully been able to do previously.
        \item Ester, cross-Claisen, hydrolysis/decarboxyliation could also allow you to make a series of different imidazoles.
    \end{itemize}
    \item Example synthesis: pan-JAK inhibitor.
    \begin{itemize}
        \item Lab synthesis.
        \begin{itemize}
            \item Buy the phenol and protect it as the SEM.
            \item Miyaura borylation, Suzuki-Miyaura coupling, Pinner salt formation, convert to the imidium system.
            \item Cleave with acid to liberate the carbonyl and do the intramolecular cyclization.
        \end{itemize}
        \item Scale synthesis.
        \begin{itemize}
            \item Removed Miyaura borylation with Grignard, etc.
            \item Gets a byproduct, but it's inactive.
            \item Many telescoped steps.
            \item You need to worry about the form of the crystal that recrystallizes (there is a whole field of \textbf{crystal engineering}); is it too big, too small, etc.?
        \end{itemize}
    \end{itemize}
    \item Example synthesis: P13K $\beta$-Sparing.
    \begin{itemize}
        \item No lateral deprotonation, despite intuition!
        \item Weinreb amide (for adding carbanions to carbonyl derivatives).
        \item Alkylate on nitrogen, do S\textsubscript{N}Ar (could also be benzyne).
        \item Palladium catalyst for final Suzuki-Miyaura cross-coupling.
        \begin{itemize}
            \item Can do it in the presence of \emph{a lot} of basic functional groups.
            \item More evidence why this chemistry won the Nobel prize.
        \end{itemize}
    \end{itemize}
    \item 1,2-azoles.
    \begin{itemize}
        \item We'll talk mostly about pyrazole, but there's also isothiazole and isoxazole.
        \item Dimeric structure in solution.
        \item Also has tautomerization.
    \end{itemize}
    \item A few reactions (similar again).
    \begin{itemize}
        \item N lone pair in and out of the aromatic system.
        \item Acylation $\to$ deprotonation again.
        \item Selective halogenation can be rationalized based on arrow-pushing and charges.
        \item Can acylate on carbon by sterically blocking the site that will typically react first; thus, more engineered and less useful.
        \item Under neutral conditions, alkylation occurs at the lone pair.
        \begin{itemize}
            \item Under basic conditions, we form the thermodynamic product.
        \end{itemize}
        \item Lots of companies have wanted to \emph{N}-arylate at the thermodynamically unfavored nitrogen recently, and have needed catalysts to do that.
        \item Lithiation.
    \end{itemize}
    \item Pyrazole syntheses.
    \begin{itemize}
        \item More condensation chemistry.
        \begin{itemize}
            \item Always look for bisnucleophiles and biselectrophiles!
            \item This is a very common disconnection.
        \end{itemize}
        \item Dipolar cycloadditions can also be employed (not as common, but occur on occasion).
        \item Knorr gets his own synthesis.
        \begin{itemize}
            \item This is good for symmetric pyrazoles.
        \end{itemize}
        \item Propynyl ketones act as the synthetic equivalent of a $\beta$-dicarbonyl.
        \item Cyclopropane thing synthesis.
        \begin{itemize}
            \item Take the diketone, halogenate in between, nucleophilic displacement. $\alpha$-aryloxy ketone could feed into a cross-Claisen condensation.
        \end{itemize}
        \item Aside: Whenever you see a structure, think about whether you can get to it using chemistry that you learned in first-year organic; that's what people want to use.
        \item Diazomethane can be generated in flow now, so it can be used on scale.
        \begin{itemize}
            \item Explosive and toxic; precursors are also nasty (mutagenic), so bad on lab scale, too.
        \end{itemize}
        \item What if the condensation has $2:1$ selectivity in the wrong direction?
        \begin{itemize}
            \item Try a dipolar cycloaddition.
            \item Treat a thing with base to do a 1,3-elimination. Then do this with an aryl acetylene (looks good, but hard to handle and explosive, so use an equivalent).
            \item As an equivalent, use the enamine, which is an elimination away from the acetylene.
            \item They did this chemistry on a huge scale, which is wild to Steve.
        \end{itemize}
        \item In process chemistry, they will do almost anything (as long as its legal), even using brutal conditions, if necessary.
    \end{itemize}
    \item DGAT-2.
    \begin{itemize}
        \item Cyclopropanated benzimidazole derivative.
        \item Reduce to the 1,2-diamino derivative. Then other piece for condensation.
        \item Other piece: $\alpha$-alkylation twice. Can't do S\textsubscript{N}2 with cyclopropanes because the transition state wants to be \ang{120}, but the cyclopropane is \ang{60}.
        \item GMP (General Manufacturing Procedure) synthesis (control access to the reactors, everyone is in clean suits, etc.). Very expensive, but makes sense if the compound is going into a person.
        \item Got starting material from $\gamma$-bromocarboxylic acid via Hell-Volhard-Zelinsky reaction, in Steve's guess.
        \item Cleave the ester under acidic conditions; in basic, you would have competitive S\textsubscript{N}Ar?? (easier to control the quality of acetyl chloride and methanol than gaseous chloride, so as to generate \ce{HCl} \emph{in situ}).
        \item Do this in the presence of Boc-anhydride to form the Boc-amide.
        \item Use T3P (a reagent to make amides).
        \item Then cleave the Boc.
    \end{itemize}
    \item Indoles.
    \begin{itemize}
        \item Jeremy Knowles (Steve's doctoral advisor) used to make fun of people who made indoles, yet Steve ended up making them regardless.
        \item Most widely occuring ones: (\emph{S})-tryptophan and seratonin (responsible for sleep, depression, anxiety, etc.).
        \item SSRIs: Selective seratonin uptake inhibitors.
        \begin{itemize}
            \item Triptans are antimigrane drugs, very structurally related to seratonin.
            \item Migraines are financially huge to pharmaceutical companies. No generally successful solutions yet.
        \end{itemize}
        \item LSD.
    \end{itemize}
    \item Reactions of indoles.
    \begin{itemize}
        \item 5-membered ring is always the most reactive part.
        \item \SI{6}{\molar} sulfuric acid reveals that protonation at C3 is most favorable.
        \begin{itemize}
            \item $\pKa=-3.5$, so does not protonate easily.
        \end{itemize}
        \item React with electrophiles at C3.
        \begin{itemize}
            \item Example: Halogenation occurs at C3.
        \end{itemize}
        \item Acylation.
        \begin{itemize}
            \item Acidic conditions: C3.
            \item Basic conditions: At the nitrogen.
            \item C3-blocking leads to C2 reactivity next.
        \end{itemize}
        \item Excess of methyl iodide and heat leads to tetramethylated isoindole structure. Write how this forms!!
        \begin{itemize}
            \item Skatole (one of the worst smelling compounds in the world) is the product; look it up!
        \end{itemize}
        \item \ce{BF3}-etherate.
        \begin{itemize}
            \item Proceeds through spirocyclic intermediate (very common chemistry for indoles), as proven by isotopic labeling.
            \item Aside: On mechanisms.
            \begin{itemize}
                \item You used to have cumulative exams and 2 foreign languages as PhD requirements.
                \item Frank Westheimer (famous guy who invented chemical biology) was one of Steve's "cumes." One question he gave was "cite the original experimental evidence for these 20 famous findings;" Steve had no idea.
            \end{itemize}
        \end{itemize}
        \item Mannich-type reactions.
        \begin{itemize}
            \item $\pH=6$ is the Goldilocks range.
            \item Pictet-Spengler type transformation, historically used in alkaloid synthesis.
        \end{itemize}
        \item With base.
        \begin{itemize}
            \item NaH is fine, but not great on scale (usually shipped as mineral oil dispersion).
            \item EtMgI is shipped around in tank cars and it forms a base just fine.
        \end{itemize}
        \item Directed metallation.
        \begin{itemize}
            \item BOC is DMG, then deprotonate at C2, then electrophile.
            \item Cooler way: Throw dry ice in (\ce{CO2} source). Treat with more to form dianion, then deprotect.
        \end{itemize}
    \end{itemize}
    \item Reactions of gramine.
    \begin{itemize}
        \item Tryptophan.
        \begin{itemize}
            \item Put something on that isn't a great leaving group.
            \item Put on an electron conduit that allows you to push out bad leaving groups.
            \item This is a way to make racemic tryptophan.
        \end{itemize}
        \item N-methylation.
        \begin{itemize}
            \item TIPS (big) allows for C4 lithiation.
            \item This is important because the \textbf{Fischer indole syntehsis} (typical) is not good at making 4-substituted indoles.
        \end{itemize}
    \end{itemize}
    \item We'll start with indole synthesis next time.
    \item Next Tuesday, after class: PSet 2 and previous years' exams.
\end{itemize}




\end{document}