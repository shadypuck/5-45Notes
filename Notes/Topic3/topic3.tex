\documentclass[../notes.tex]{subfiles}

\pagestyle{main}
\renewcommand{\chaptermark}[1]{\markboth{\chaptername\ \thechapter\ (#1)}{}}
\setcounter{chapter}{2}

\begin{document}




\chapter{\texorpdfstring{$\bm{\pi}$}{TEXT}-Excessive Heterocycles}
\section{Imidazoles, Pyrazoles, and Indoles}
\begin{itemize}
    \item \marginnote{2/20:}Announcements.
    \begin{itemize}
        \item PSet 2 will not be posted yet because most of the material won't be covered until next Tuesday.
        \begin{itemize}
            \item We'll still have it a week before the exam, and the exam will not be so indole focused.
        \end{itemize}
        \item The practice exams are also still to come.
        \item Lots of material and esoteric reactions in these slides; Steve will not discuss it all, nor expect that we remember it all.
    \end{itemize}
    \item \textbf{Imidazole} and \textbf{benzimidazole}.
    \begin{itemize}
        \item Important constituents in pharmaceuticals and biologically important substructures (e.g., histidine; nucleophile in salt bridges; constituent in DNA).
        \item Proteins are often purified on nickel columns that act on histidines (see "HisTags").
    \end{itemize}
    \item Structure and reactivity.
    \begin{itemize}
        \item Often put into structures to increase water solubility (can completely hydrogen-bond; both donor and acceptor)!
        \item Combination of pyridine and pyrrole: One lone pair orthogonal to the $\pi$-system, and one pyrrole-like pair that does not typically react with electrophiles.
        \item Imidazole is less nucleophilic than pyrrole at carbon.
        \item Rapid tautomerization complicates reactivity; if you want to target a particular site, you might get a surprise. But there are ways to overcome this that we'll discuss.
        \item Imidazole is amphoteric: One \ce{H} is moderately acidic (not super, but not like \ce{C-H} either), and then can protonate. Much less acidic than oxazole or thiazole because of resonance.
        \item Alkylation at nitrogen occurs, followed by deprotonation, followed by more reactivivity.
    \end{itemize}
    \item Reactions of imidazole.
    \begin{itemize}
        \item Deprotonation (with a strong base, e.g., \ce{NaH}, LDA, \ce{NaHMDS}) creates a strong base that monoalkylates.
        \item Selective alkylation at nitrogen?
        \begin{itemize}
            \item Target \ce{N^2} with protection (acylation), \textbf{ethyl Meerwein's reagent}, and deacylation.
            \item Target \ce{N^1} with Buchwald amination (Steve: "I hate the amination, should've gotten rid of the amination").
        \end{itemize}
        \item EAS.
        \begin{itemize}
            \item Better than benzene, worse than pyrrole.
            \item Nitration breaks the symmetry of the molecule. Easier to put nitrogen group next to non-positively charged nitrogen.
            \item Polybromination is also possible.
            \item Selective bromination occurs analogously to with pyridine (see Figure \ref{fig:TTQPy3Br}); attack at nitrogen, then carbon, then rearomatization.
            \item More "trivial" reactions.
        \end{itemize}
        \item S\textsubscript{N}Ar.
        \begin{itemize}
            \item Moves the lone pair onto the nitrogen, as we've seen.
        \end{itemize}
        \item Directed metallation.
        \begin{itemize}
            \item SEM (popularized by Bruce Lipschutz at UC-Santa Barbara) is the best protecting group. Can be removed by fluoride, which induces a loss of ethylene and formaldehyde.
            \item Selective deprotonation between the two nitrogens (fairly standard, steric factors considered).
            \item Can then do again.
        \end{itemize}
        \item Lithium-halogen exchange.
        \begin{itemize}
            \item LiX exchange occurs faster than deprotonation, then deprotonation occurs.
            \item To ensure that everything occurs in the right order, people will often add a strong base (e.g., \ce{LiHMDS}) first; then add butyl lithium to do the LiX exchange.
        \end{itemize}
        \item Radical chemistry, e.g., the Minisci reaction.
        \begin{itemize}
            \item Photochemistry as well, but that would be a whole other course; Steve won't discuss, take 5.44 with Alison if you want to hear more.
            \item Minisci (radical decarboxylation) predates photoredox catalysis for generation of carbon-centered radicals.
            \item Second example: Nucleophilic radical can add to electron-deficient (because of the aldehyde) carbon center.
        \end{itemize}
        \item Quaternary imidazolium salts.
        \begin{itemize}
            \item Subsequent base yields ylide, i.e., the NHC (NHCs ubiquitous in catalysis).
        \end{itemize}
    \end{itemize}
    \item Selected imidazole disconnections.
    \begin{itemize}
        \item Some should look familiar, and some may not.
        \item The first one to talk about individually is the \textbf{Debus-Radziszewski} (imidazole synthesis).
        \begin{itemize}
            \item From a long, long time ago. First reported synthesis of imidazole.
            \item Combines a 1,2-diketone, aldehyde, and ammonia.
            \item Proposed mechanism has zero evidence, but some variation is probably correct.
            \begin{itemize}
                \item Aldehyde is probably converted to imine \emph{before} formation of the diimine on the 1,2-diketone.
                \item Then condensation.
                \item Then tautomerization.
            \end{itemize}
        \end{itemize}
        \item Synthesis 1: Analogous to the Pinner reaction; very common.
        \item Van Leusen: Analogous to the pyrrole synthesis of the same name.
        \item Synthesis 4.
        \begin{itemize}
            \item Aminoacetal an acetal for stability reasons.
            \item Attack to imine and then cyclization.
        \end{itemize}
        \item Synthesis 6: Paal-Knorr type.
    \end{itemize}
    \item Example synthesis: Conivaptan.
    \begin{itemize}
        \item Pinner-type synthesis.
    \end{itemize}
    \item Example synthesis: Estrogen receptor.
    \begin{itemize}
        \item \ce{N-} adds to nitrile.
        \item Workup to amidine.
        \item Condense with $\alpha$-bromoacetaldehyde to form the imidazole.
    \end{itemize}
    \item Example synthesis: Obesity.
    \begin{itemize}
        \item $\alpha$-bromopyruvate.
    \end{itemize}
    \item Van Leusen.
    \begin{itemize}
        \item TosMIC: Stabilized isocyanide.
        \item Easily deprotonated, add to the imine, attack at carbene, proton transfer, losing the sulfonate.
    \end{itemize}
    \item COX-2 inhibitors.
    \begin{itemize}
        \item Historically important chemistry.
        \begin{itemize}
            \item Merck billion-dollar molecule.
            \item Has to do with pain.
            \item Aspirin (but disrupts stomach) $\to$ NSAIDs (ibuprofen, endoproxin) $\to$ tylenol (but dissolves liver) $\to$ opioids (but addictive).
            \item Most things inhibit both pathways (COX-1 and COX-2), but this drug was selective for COX-2, specifically. But this (Vioxx) causes heart-valve problems (and Merck had to pull it from the market at great loss).
            \item Celebrex as well, but the company died and had to be sold to Pfizer.
        \end{itemize}
    \end{itemize}
    \item Example synthesis: Like Lipitor, another statin compound.
    \begin{itemize}
        \item Glycine benzyl ester is a fairly standard protected amino acid.
        \item Treat it to form something, which after cleavage can be acylated.
        \item Ester to benzylamine.
        \item Cyclize with a primary amine to stitch in the nitrogen.
        \item Cyclize (fairly typical with statins).
    \end{itemize}
    \item Example synthesis: Debus-Radziszewski chemistry.
    \begin{itemize}
        \item Microwave chemistry was huge, but the bubble has burst at this point. You still see it here and there, but not much.
        \item Get to the asymmetric $\alpha$-diketone with a \ce{SeO2} oxidation.
        \item What method you'll use commonly depends on what you have and what you have successfully been able to do previously.
        \item Ester, cross-Claisen, hydrolysis/decarboxyliation could also allow you to make a series of different imidazoles.
    \end{itemize}
    \item Example synthesis: pan-JAK inhibitor.
    \begin{itemize}
        \item Lab synthesis.
        \begin{itemize}
            \item Buy the phenol and protect it as the SEM.
            \item Miyaura borylation, Suzuki-Miyaura coupling, Pinner salt formation, convert to the imidium system.
            \item Cleave with acid to liberate the carbonyl and do the intramolecular cyclization.
        \end{itemize}
        \item Scale synthesis.
        \begin{itemize}
            \item Removed Miyaura borylation with Grignard, etc.
            \item Gets a byproduct, but it's inactive.
            \item Many telescoped steps.
            \item You need to worry about the form of the crystal that recrystallizes (there is a whole field of \textbf{crystal engineering}); is it too big, too small, etc.?
        \end{itemize}
    \end{itemize}
    \item Example synthesis: P13K $\beta$-Sparing.
    \begin{itemize}
        \item No lateral deprotonation, despite intuition!
        \item Weinreb amide (for adding carbanions to carbonyl derivatives).
        \item Alkylate on nitrogen, do S\textsubscript{N}Ar (could also be benzyne).
        \item Palladium catalyst for final Suzuki-Miyaura cross-coupling.
        \begin{itemize}
            \item Can do it in the presence of \emph{a lot} of basic functional groups.
            \item More evidence why this chemistry won the Nobel prize.
        \end{itemize}
    \end{itemize}
    \item 1,2-azoles.
    \begin{itemize}
        \item We'll talk mostly about pyrazole, but there's also isothiazole and isoxazole.
        \item Dimeric structure in solution.
        \item Also has tautomerization.
    \end{itemize}
    \item A few reactions (similar again).
    \begin{itemize}
        \item N lone pair in and out of the aromatic system.
        \item Acylation $\to$ deprotonation again.
        \item Selective halogenation can be rationalized based on arrow-pushing and charges.
        \item Can acylate on carbon by sterically blocking the site that will typically react first; thus, more engineered and less useful.
        \item Under neutral conditions, alkylation occurs at the lone pair.
        \begin{itemize}
            \item Under basic conditions, we form the thermodynamic product.
        \end{itemize}
        \item Lots of companies have wanted to \emph{N}-arylate at the thermodynamically unfavored nitrogen recently, and have needed catalysts to do that.
        \item Lithiation.
    \end{itemize}
    \item Pyrazole syntheses.
    \begin{itemize}
        \item More condensation chemistry.
        \begin{itemize}
            \item Always look for bisnucleophiles and biselectrophiles!
            \item This is a very common disconnection.
        \end{itemize}
        \item Dipolar cycloadditions can also be employed (not as common, but occur on occasion).
        \item Knorr gets his own synthesis.
        \begin{itemize}
            \item This is good for symmetric pyrazoles.
        \end{itemize}
        \item Propynyl ketones act as the synthetic equivalent of a $\beta$-dicarbonyl.
        \item Cyclopropane thing synthesis.
        \begin{itemize}
            \item Take the diketone, halogenate in between, nucleophilic displacement. $\alpha$-aryloxy ketone could feed into a cross-Claisen condensation.
        \end{itemize}
        \item Aside: Whenever you see a structure, think about whether you can get to it using chemistry that you learned in first-year organic; that's what people want to use.
        \item Diazomethane can be generated in flow now, so it can be used on scale.
        \begin{itemize}
            \item Explosive and toxic; precursors are also nasty (mutagenic), so bad on lab scale, too.
        \end{itemize}
        \item What if the condensation has $2:1$ selectivity in the wrong direction?
        \begin{itemize}
            \item Try a dipolar cycloaddition.
            \item Treat a thing with base to do a 1,3-elimination. Then do this with an aryl acetylene (looks good, but hard to handle and explosive, so use an equivalent).
            \item As an equivalent, use the enamine, which is an elimination away from the acetylene.
            \item They did this chemistry on a huge scale, which is wild to Steve.
        \end{itemize}
        \item In process chemistry, they will do almost anything (as long as its legal), even using brutal conditions, if necessary.
    \end{itemize}
    \item DGAT-2.
    \begin{itemize}
        \item Cyclopropanated benzimidazole derivative.
        \item Reduce to the 1,2-diamino derivative. Then other piece for condensation.
        \item Other piece: $\alpha$-alkylation twice. Can't do S\textsubscript{N}2 with cyclopropanes because the transition state wants to be \ang{120}, but the cyclopropane is \ang{60}.
        \item GMP (General Manufacturing Procedure) synthesis (control access to the reactors, everyone is in clean suits, etc.). Very expensive, but makes sense if the compound is going into a person.
        \item Got starting material from $\gamma$-bromocarboxylic acid via \textbf{Hell-Volhard-Zelinsky} reaction, in Steve's guess.
        \item Cleave the ester under acidic conditions; in basic, you would have competitive S\textsubscript{N}Ar?? (easier to control the quality of acetyl chloride and methanol than gaseous chloride, so as to generate \ce{HCl} \emph{in situ}).
        \item Do this in the presence of Boc-anhydride to form the Boc-amide.
        \item Use T3P (a reagent to make amides).
        \item Then cleave the Boc.
    \end{itemize}
    \item Indoles.
    \begin{itemize}
        \item Jeremy Knowles (Steve's doctoral advisor) used to make fun of people who made indoles, yet Steve ended up making them regardless.
        \item Most widely occuring ones: (\emph{S})-tryptophan and seratonin (responsible for sleep, depression, anxiety, etc.).
        \item SSRIs: Selective seratonin uptake inhibitors.
        \begin{itemize}
            \item Triptans are antimigrane drugs, very structurally related to seratonin.
            \item Migraines are financially huge to pharmaceutical companies. No generally successful solutions yet.
        \end{itemize}
        \item LSD.
    \end{itemize}
    \item Reactions of indoles.
    \begin{itemize}
        \item 5-membered ring is always the most reactive part.
        \item \SI{6}{\molar} sulfuric acid reveals that protonation at C3 is most favorable.
        \begin{itemize}
            \item $\pKa=-3.5$, so does not protonate easily.
        \end{itemize}
        \item React with electrophiles at C3.
        \begin{itemize}
            \item Example: Halogenation occurs at C3.
        \end{itemize}
        \item Acylation.
        \begin{itemize}
            \item Acidic conditions: C3.
            \item Basic conditions: At the nitrogen.
            \item C3-blocking leads to C2 reactivity next.
        \end{itemize}
        \item Excess of methyl iodide and heat leads to tetramethylated isoindole structure. Write how this forms!!
        \begin{itemize}
            \item Skatole (one of the worst smelling compounds in the world) is the product; look it up!
        \end{itemize}
        \item \ce{BF3}-etherate.
        \begin{itemize}
            \item Proceeds through spirocyclic intermediate (very common chemistry for indoles), as proven by isotopic labeling.
            \item Aside: On mechanisms.
            \begin{itemize}
                \item You used to have cumulative exams and 2 foreign languages as PhD requirements.
                \item Frank Westheimer (famous guy who invented chemical biology) was one of Steve's "cumes." One question he gave was "cite the original experimental evidence for these 20 famous findings;" Steve had no idea.
            \end{itemize}
        \end{itemize}
        \item Mannich-type reactions.
        \begin{itemize}
            \item $\pH=6$ is the Goldilocks range.
            \item Pictet-Spengler type transformation, historically used in alkaloid synthesis.
        \end{itemize}
        \item With base.
        \begin{itemize}
            \item NaH is fine, but not great on scale (usually shipped as mineral oil dispersion).
            \item EtMgI is shipped around in tank cars and it forms a base just fine.
        \end{itemize}
        \item Directed metallation.
        \begin{itemize}
            \item BOC is DMG, then deprotonate at C2, then electrophile.
            \item Cooler way: Throw dry ice in (\ce{CO2} source). Treat with more to form dianion, then deprotect.
        \end{itemize}
    \end{itemize}
    \item Reactions of gramine.
    \begin{itemize}
        \item Tryptophan.
        \begin{itemize}
            \item Put something on that isn't a great leaving group.
            \item Put on an electron conduit that allows you to push out bad leaving groups.
            \item This is a way to make racemic tryptophan.
        \end{itemize}
        \item N-methylation.
        \begin{itemize}
            \item TIPS (big) allows for C4 lithiation.
            \item This is important because the \textbf{Fischer indole syntehsis} (typical) is not good at making 4-substituted indoles.
        \end{itemize}
    \end{itemize}
    \item We'll start with indole synthesis next time.
    \item Next Tuesday, after class: PSet 2 and previous years' exams.
\end{itemize}



\section{Indoles, Indazoles, Thiophenes, and Furans}
\begin{itemize}
    \item \marginnote{2/25:}Announcements.
    \begin{itemize}
        \item Exam 1: Next Tuesday, in class, same format as the two previous exams.
        \begin{itemize}
            \item Purpose: Steve is required to give one.
            \item Confirm what you know; have you paid attention, stayed awake, etc.? Some regurgitation.
            \item What can you do with the material you know? Arrow pushing, etc. More like mechanistic problems.
            \item A few synthesis problems.
            \item Some aspects of metal-catalyzed cross-coupling. You don't need to know this ligand vs. that, but you should know the basic features of \ce{C-C} cross-coupling, basic steps of the reaction, what metal works, know some ligand, etc.
            \item The exam will be \emph{distinct} from the previous exams.
            \item Current difficulty (before the TA edits it): Moderate.
            \item Today's lecture material is the end of what will be covered on the exam.
        \end{itemize}
        \item PSet 2 is much more indole-focused than the exam.
    \end{itemize}
    \item Synthesis of indoles.
    \item \textbf{Bartoli} (indole synthesis): Vinyl grignard plus nitroarene.
    \begin{itemize}
        \item You have to believe it was discovered by accident, because it makes so little sense.
        \item You need a relatively large \ce{R} group (bromine counts as relatively large).
        \item You can write a mechanism (this is plausible, but it may or may not have any basis in reality).
        \item Plausible mechanism: Nucleophilic attack at oxygen, collapse to a nitroso intermediate, nucleophilic attack, sigmatropic rearrangement, intramolecular attack, deprotonation and rearomatization, and then workup.
        \item Example: Propenyl grignard gives 3-methyl substituted.
        \item Indole's 7-position is not trivial to functionalize, so having a starting material with that position activated that you can then Heck couple to later (or do something else to) is super useful.
    \end{itemize}
    \item Now some more historically important indole syntheses.
    \item \textbf{Leimgruber-Batcho} (indole synthesis): Mix \emph{ortho}-alkylated nitroarene with Brederech's reagent,\footnote{It appears that this is not actually "Brederech's reagent," but DMF-dimethylacetal.} and then heat it in DMF.
    \begin{itemize}
        \item Mechanism: Spontaneously generates a bit of methoxide to do lateral deprotonation. Then addition to the compound formed by expulsion of the methoxide. This gives enamine. Now magic chemistry: Reduce nitro group to an amine, then addition-elimination to indole.
        \item It's not been carefully elucidated what does the reduction, but the guess is that using "tickle 3" (\ce{TiCl3}) does inner sphere addition to nitroso, reduction of the nitroso, etc. Not yet published what actually happens.
    \end{itemize}
    \item \textbf{Bischler} (indole synthesis): Mix an aniline with an $\alpha$-bromoacetaldehyde acetal.
    \begin{itemize}
        \item A base deprotonates the aniline, which then engages in S\textsubscript{N}2 bromide displacement.
        \item Then, adding trifluoroacetic anhydride (TFAA) forms the \emph{N}-trifluoromethylacetal.
        \item Trifluoroacetyl groups are very labile. Acetamides are often the bane of synthetic chemists (very hard to cleave), but trifluoroacetamides are much easier to cleave (sometimes too easy).
        \item Stabilized oxocarbenium then does Friedel-Crafts type chemistry.
    \end{itemize}
    \item Protecting groups in general tend to fall off of indoles (e.g., Bocs, etc.). This is why you often have to resort to using a SEM, but those can be difficult to remove.
    \item \textbf{Fischer} (indole synthesis): Mix an aryl hydrazine with a ketone.
    \begin{itemize}
        \item Most important.
        \item Also had to be discovered by accident. Here are Steve's thoughts on its origin.
        \begin{itemize}
            \item Before NMR and IR, you had EA and melting point only. You determined molecular structure by making derivatives of certain functional groups and then taking melting points.
            \item For example, \textbf{Tollens' reagent} (silver-based) was used to figure out if there was an aldehyde.
            \item As another example, diphenylhydrazine was used to make a hydrazone. Hydrazones are super crystalline, so it's easy to get their melting point.
            \item They were probably making a derivative, then realized that they made an indole!
        \end{itemize}
        \item Mechanism: Condensation to the aryl hydrazone, tautomerization to \textbf{ene-hydrazine}, $[3,3]$-sigmatropic rearrangement, rearomatization. Then ene or iminium formation.
        \begin{itemize}
            \item They did not know what sigmatropics were back then, so that definitely just happened.
            \item To make an aryl hydrazine, you make the aryl diazonium salt and then reduce it (typically with \ce{SnCl2}).
        \end{itemize}
        \item Limitations.
        \begin{itemize}
            \item If \ce{R} and \ce{R$'$} are distinct, then the first intermediate can enolize two different ways, which leads to regioisomer formation.
            \item Substituents at 4- or 6-positions on the aromatic ring lead to ambiguity in where the sigmatropic rearrangement can occur.
            \item Forcing conditions (strong acid and heating) can lead to issues with sensitive functional groups (esp. aldehydes).
        \end{itemize}
        \item You can manipulate the system, though.
        \begin{itemize}
            \item Stronger vs. weaker acids modulate the direction of enolization. Kinetic vs. thermodynamic character; thermodynamic with the stronger acids.
        \end{itemize}
        \item Limitations are important to know because you want to know the plusses and minuses of each method.
    \end{itemize}
    \item \textbf{Reissert} (indole synthesis): \emph{ortho}-alkylated nitroarene, again, plus an oxalate.
    \begin{itemize}
        \item Strong base leads to lateral deprotonation, addition to $\alpha$-ketoester, then reduce to form the 2-ethylcarboxylate of indole.
        \item Can then do addition at 3-position to form differentially substituted 2,3-disubstituted indole.
    \end{itemize}
    \item \textbf{Madelung} (indole synthesis): \emph{N}-\emph{ortho}-alkylarylamide collapses in strong base.
    \begin{itemize}
        \item Deprotonation, probably via the dianion, which closes to form the indole and then can be further modified.
    \end{itemize}
    \item \textbf{Hemetsberger} (indole synthesis): Collapse of an $\alpha$-azidoester on a styrene-type thing.
    \begin{itemize}
        \item Fancier and less safe.
        \item The starting material can be made from benzaldehyde and \ce{R}-azidoacetate via Knoevenagel.
        \item Mechanism: Photolyse to nitrene, which rearranges to azirine, which rearranges to the indole derivative.
    \end{itemize}
    \item Example synthesis: Applying the Leimgruber-Batcho indole synthesis.
    \begin{itemize}
        \item Introduce two sulfonyl protecting groups so that you can put the nitro group at the desired position.
        \item Superheated steam is a classic way to do desulfonization.
        \item Benzyl-protect the phenol group.
        \item Do Leimgruber-Batcho.
        \begin{itemize}
            \item A pyrrolidine enamine is fairly common in this reaction.
            \item Then convert to \textbf{semicarbazide}, to crystallize/isolate the intermediate before proceeding.
            \item Now some 21st century chemistry: Reduce the nitro group and other functional group with iron under acidic conditions.
        \end{itemize}
        \item Then you add the aniline to the imine to form the aminal-type molecule, and collapse.
    \end{itemize}
    \item Example synthesis: Applying the Hemetsberger synthesis.
    \begin{itemize}
        \item Not often used because of "azidophobia."
        \item Reduce and oxidize to make the aldehyde.
        \item Knoevenagel condensation to Hemetsberger starting material.
        \item Reflux to complete the synthesis.
        \item Hydrolyze the ester to the acid, make the primary amide, and then dehydrate to the nitrile.
        \begin{itemize}
            \item Catalytic DMF with oxalyl chloride forms the Vilsmeier reagent, which can then chlorinate carboxylic acids before amidation.
            \item Whereas \ce{POCl3} forms the Vilsmeier reagent via the enthalpic driving force of strong \ce{P=O} bond formation, \ce{(COCl)2} forms the Vilsmeier reagent via the entropic driving force of \ce{CO2 + CO} gas release.
        \end{itemize}
        \item Deprotonate with \ce{KH} first because if you don't, you might get reduction of the lithium/halogen-exchanged species.
        \begin{itemize}
            \item Essentially, pre-deprotonation allows us to reliably and quantitatively form the dianion, whereas if we go straight through 2 eq. \ce{{}^{\emph{n}}BuLi}, we'll do LiX exchange first (kinetically faster) and then the anion will deprotonate the \ce{N-H}. The result is that we'll have significant dehalogenated side product.
        \end{itemize}
        \item Then we add the anion into DMF, and warm/acidify to collapse.
    \end{itemize}
    \item In industry, they do tons of safety evaluations (both for safety and because blowing up a reactor is expensive).
    \begin{itemize}
        \item You want your calorimetry to give you \SI{80}{\celsius} between your reaction temperature and the exotherm.
        \begin{itemize}
            \item That way, no part of the mixture is likely to get hot enough to induce a runaway reaction.
        \end{itemize}
        \item This is another example of the use of flow chemistry (it can control thermal runaways).
    \end{itemize}
    \item Indazoles.
    \begin{itemize}
        \item There exist 1- and 2-indazoles.
        \item 1-indazoles are more common.
    \end{itemize}
    \item Reactivity of indazoles.
    \begin{itemize}
        \item \emph{N}-substitution/protection.
        \begin{itemize}
            \item Under basic conditions, bonding at either nitrogen is equally likely.
            \item Under acidic conditions, N2-THP substitution occurs more quickly but N1-THP substitution is more thermodynaically favorable.\footnote{\href{https://www.researchgate.net/figure/Scheme-1-Protection-of-an-alcohol-by-Thp-and-its-elimination-mechanism_fig2_314357461}{Mechanism}.}
            \item Thus, strong acid gives exclusively N1-THP substitution while weak acid is more likely to give N2-THP substitution (or a mixture at long reaction times).
        \end{itemize}
        \item Palladium catalyzed \ce{C-C} or \ce{C-N} coupling.
        \item It's often necessary to protect a nitrogen first.
    \end{itemize}
    \item Syntheses of indazoles.
    \begin{itemize}
        \item Route 1: Start with an \emph{ortho}-alkylated aniline, acidify, form diazonium, do lateral deprotonation and collapse.
        \item Route 2: Start with an \textbf{isatin}, and then use diazonium conditions again.
        \begin{itemize}
            \item Isatins show up not infrequently in the literature.
            \item Isatins can be made from anilines and chloral,\footnote{\href{https://synarchive.com/named-reactions/sandmeyer-isatin-synthesis}{Mechanism}.} then hydroxylamine, then strong acid can also be good.
        \end{itemize}
        \item Route 3: Start with bromofluorobenzaldehyde.
        \begin{itemize}
            \item In S\textsubscript{N}Ar, it's the electron-withdrawing nature of the substituent that's important for selectivity (so \ce{F-} is a better leaving group!).
            \item Hydrazone formation first, and then intramolecular S\textsubscript{N}Ar.
            \item With nitrile or ester SM, you get different 3-substituted indazoles.
        \end{itemize}
    \end{itemize}
    \item Example synthesis: EGFR kinase inhibitor.
    \begin{itemize}
        \item Protect with THP, use Xantphos (a great ligand for ??) to do \ce{C-N} coupling.
        \item Acrylamide inhibitor makes this another covalent inhibitor.
    \end{itemize}
    \item Moving on back to something.
    \item Comparing $\pi$-excessive heterocycles: Structure.
    \begin{itemize}
        \item Furan is least aromatic, then pyrrole, then thiophene is most aromatic.
        \begin{itemize}
            \item All have one lone pair in aromatic system.
        \end{itemize}
        \item Furan is more reactive; lower cost to dearomatize.
        \item Aromaticity trends are in accord with electronegativity of heteroatom (more electronegativity means less willing to delocalize).
    \end{itemize}
    \item Comparing $\pi$-excessive heterocycles: Relative rates of acylation with TFAA.
    \begin{itemize}
        \item Enormous reactivity difference: Pyrrole much more reactive than furan, more reactive than thiophene, and benzene doesn't react.
        \item Selectivity.
        \begin{itemize}
            \item $\alpha$-addition is preferred because you get resonance delocalization through the $\pi$-system as well.
        \end{itemize}
    \end{itemize}
    \item Thiophenes.
    \begin{itemize}
        \item Thiophene, benzothiophene = benzo[b]thiophene, and benzo[c]thiophene.
        \item Derived from two Greek words: Sulfur and shining.
        \item Discovered as a contaminant in benzene.
        \begin{itemize}
            \item Benzene used to be sold as "thiophene-free." If you were doing electrophilic reactions, thiophene was more reactive so you would get contaminants derived from it.
        \end{itemize}
    \end{itemize}
    \item Reactivity of thiophene.
    \begin{itemize}
        \item Tetrabromothiophene can be made; tetraiodothiophene can't be made (iodines are too big).
        \item Selective reduction can be done with palladium and \ce{NaBH4}: Oxidative addition is better at the $\alpha$-position, and one $\alpha$ is much less hindered than the other.
    \end{itemize}
    \item Syntheses: The usual suspects (Paal-Knorr), and then some other reactions (Hinsberg, Gewald [pretty useful], Fiesslemann).
    \item Commercial synthesis of thiophene.
    \begin{itemize}
        \item Butane and elemental sulfur, with a catalyst at \SI{600}{\celsius}.
        \item Another commercial route: Butanol and carbon disulfide.
    \end{itemize}
    \pagebreak
    \item \textbf{Paal-Knorr} (thiophene synthesis): Heteroatom nucleophile and 1,4-diketone.
    \begin{itemize}
        \item Example heteroatom nucleophiles: \ce{H2S + HCl}, \ce{P4S10}, Lawesson's reagent.
        \item Lawesson's reagent does sulfur Wittigs on a carbonyl: \ce{C=O} to \ce{C=S}. Driving force is strong \ce{P=O} bond formation.
    \end{itemize}
    \item \textbf{Fiesselmann} (thiophene synthesis).
    \begin{itemize}
        \item $\beta$-chloroenal comes from Vilsmeier reaction; we should remember this chloroformylation!!
        \item Deprotonate, add, and dehydrate.
    \end{itemize}
    \item \textbf{Hinsberg} (thiophene synthesis).
    \begin{itemize}
        \item Related to Debus-Radziszewski in some ways.
        \item 1,2-dicarbonyl and 1,3-bisnucleophile. Deprotonate, add, eliminate twice sequentially. Then dehydrate.
        \item Heating in base leads to decarboxylation.
    \end{itemize}
    \item \textbf{Gewald} (thiophene synthesis).
    \begin{itemize}
        \item Carbonyl (usually ketone) and $\alpha$-EWG (usually cyano) cyanide.
        \item This is bucket chemistry (large scale, inexpensive reagents).
        \item Malononitrile forms dicyanoolefin, then reacts with amine to form compound shown there.
        \item Knovenagel condensation, deprotonate to form a sulfur species (you can go between \ce{S1} to \ce{S_n}, but \ce{S1} will be reactive).
        \item Then form an intermediate, followed by tautomerization.
        \item Great if you don't have regiochemical ambiguity, but can give regioisomers.
        \begin{itemize}
            \item To get around this, you cheat! Regiochemically pure alkyl bromide (raises cost), and then react.
        \end{itemize}
    \end{itemize}
    \item Example synthesis: Applying the Gewald reaction.
    \begin{itemize}
        \item Target: A weak fungicide with a silicon atom in it.
        \begin{itemize}
            \item Silicon atoms are becoming more and more common in pharmaceuticals and agrochemicals.
        \end{itemize}
        \item Discovery synthesis.
        \begin{itemize}
            \item Sandmeyer-type.
            \item \ce{{}^{\emph{n}}BuLi} for LiX exchange, then TMSCl.
            \item Acid chloride leads into amide.
        \end{itemize}
        \item This is a terrible scale synthesis, but it was "fit for purpose" (for discovery).
        \begin{itemize}
            \item Yield is bad, Sandmeyer uses a hazardous reagent, silylation at $-\SI{70}{\celsius}$, preparation of acid chloride causes 20\% protodesilylation.
        \end{itemize}
        \item Need to make rapid \SI{20}{\kilo\gram} and up to \SI{200}{\kilo\gram} batches.
        \begin{itemize}
            \item Solution: Wash out the unused material (30-40\% loss isn't environmentally good, but it can be good cost-wise).
        \end{itemize}
        \item To make even better, they went the cheating route: $\alpha$-chloro material.
    \end{itemize}
    \item Example synthesis: Ticlopedine (anti-platelet aggreation compound to lower blood pressure).
    \begin{itemize}
        \item 1st synthesis: Selectively benzylate and reduce.
        \begin{itemize}
            \item Problem: The thiophene-pyridine is not easy to access on scale.
        \end{itemize}
        \item Second synthesis: Start with thiophene-phenethyl amine, reductive amination on paraformaldehyde to cyclize, and reduce to the final product.
    \end{itemize}
    \item Skipping one.
    \item Example synthesis: Tetrasubstituted thiophenes via directed metallation.
    \begin{itemize}
        \item Cross-coupling can be good.
        \item Turbogrignard forms anion (stable at \SI{0}{\celsius} because it's thiophene; normal aromatic would eliminate to benzyne).
        \item Then selective deprotonation via DMG and treatment with ethyl cyanoformate to form diester bromide.
        \item Same base and \ce{S8} form \ce{S-} that is then alkylated.
        \item Then Miyaura borylation with \ce{Pd({}^{\emph{t}}Bu3P)2} (not \ce{Pd({}^{\emph{t}}Bu2P)2}).
        \item Then Suzuki-Miyaura cross-coupling.
    \end{itemize}
    \item Furan.
    \begin{itemize}
        \item Comes from Latin furfur (for bran), because furans come from agrochemical products; Quaker oats used to be the largest supplier of furan derivatives, esp. furfural.
        \item Least aromatic of all 5-membered heterocycles.
        \item In a variety of natural products and pharmaceuticals.
        \item Before proton pump inhibitors, we had Zantac.
    \end{itemize}
    \item Structure of furan.
    \begin{itemize}
        \item One lone pair in aromatic system.
        \item Quite acidic: $\pKa<36$ at the $\alpha$-position means we can deprotonate with \ce{{}^{\emph{n}}BuLi}.
        \item Most electron density at oxygen; a lot, as well, at the $\alpha$- and $\beta$-carbons.
        \item Industrial source: Pentose-rich matter (known as bran), then acidic hydrolysis, then dehydrate to furfural, then catalytic decarbonylation.
    \end{itemize}
    \item Reactivity of furan.
    \begin{itemize}
        \item Electrophilic reactions.
        \begin{itemize}
            \item \num{e11} times more reactive than benzene. About \SI[per-mode=symbol]{8}{\kilo\calorie\per\mole} (??) difference.
        \end{itemize}
        \item Bromination: Steve did this as a student in a poor hood and then began "bleeding profusely from [his] nose."
        \item Cycloadditions.
        \begin{itemize}
            \item Reduction and then thermolyzing (with $\Delta$) is a cute way of making the diester.
            \item Many cross-couplings.
        \end{itemize}
        \item Mannich substitution.
        \begin{itemize}
            \item Dimethylamine and formaldehyde to form iminium ion, then reacts electrophilically at the 5-position.
        \end{itemize}
        \item Achmatowicz derivative.
        \begin{itemize}
            \item Mechanism: Epoxidation of less-substituted double bond, ring-opening, then close.
        \end{itemize}
        \item Piancatelli: Don't worry about.
        \item Lithiation: Kinetic vs. thermodynamic control for 2- vs. 3-substitution.
    \end{itemize}
    \item Synthesis of furans.
    \begin{itemize}
        \item Paal-Knorr has same starting material (1,4-diketone), but is more of a rearrangement.
        \item \textbf{Feist-Benary}: Only works on aldehydes; $\alpha$-chloroketones have nucleophilic attack not at the aldehyde but at the carbon bearing chlorine.
    \end{itemize}
    \item Benzofuran = benzo[b]furan.
    \begin{itemize}
        \item First prep from Coumarin: Brominate and then treat with base to hydrolyze the lactone and decarboxylate with loss of bromine. Must have also been accidental.
    \end{itemize}
    \item More on the exam.
    \begin{itemize}
        \item You have to know stuff, but Steve also has to see that we can apply this stuff.
    \end{itemize}
\end{itemize}




\end{document}